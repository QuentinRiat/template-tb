\documentclass[
    iai, % Saisir le nom de l'institut rattaché
    mi % Saisir le nom de l'orientation
    %confidential, % Décommentez si le travail est confidentiel
]{heig-tb}

\usepackage[nooldvoltagedirection,european,americaninductors]{circuitikz}

\signature{mbernasconi.svg} % Remplacer par votre propre signature vectorielle.

\makenomenclature
\makenoidxglossaries
\makeindex

\addbibresource{bibliography.bib}

\input{nomenclature}
\input{acronyms}
\input{glossary}
% Auteur du document (étudiant-e) en projet de Bachelor
\author{Quentin Riat}

% Activer l'option pour l'accord du féminin dans le texte
\genre{male}

% Titre de votre travail de Bachelor
\title{Automatisation de la Mise en Bocaux de Légumes Lacto-fermentés}

% Le sous titre est optionnel
\subtitle{Travail de Bachelor}

% Nom du professeur responsable
\teacher {Prof. T. Maulaz (HEIG-VD)}

% Mettre à jour avec la date de rendu du travail
\date{\today}

% Numéro de TB
\thesis{-}


\surroundwithmdframed{minted}

%% Début du document
\begin{document}
\selectlanguage{french}
\maketitle
\frontmatter
\clearemptydoublepage

%% Requis par les dispositions générales des travaux de Bachelor
\preamble
\authentification

%% Résumé / Résumé publiable / Version abrégée
\begin{abstract}
    \input{abstract}
\end{abstract}

%% Sommaire et tables
\clearemptydoublepage
{
    \tableofcontents
    \let\cleardoublepage\clearpage
    \listoffigures
    \let\cleardoublepage\clearpage
    \listoftables
    \let\cleardoublepage\clearpage
    \listoflistings
}

\printnomenclature
\clearemptydoublepage
\pagenumbering{arabic}

%% Contenu
\mainmatter
\chapter{Introduction}
L'introduction est une section requise dans un rapport technique. Introduisez votre travail, l'idée de départ et les objectifs attendus. Un lecteur qui découvrirait votre projet au travers de cette introduction devrait ainsi être capable d'en comprendre le cadre, l'idée générale et les aboutissants du projet.

\section{Contexte}
Cette section \underline{n'est pas obligatoire}, mais elle est souvent présente dans un rapport technique pour compléter l'introduction et définir le contexte du travail \cad le cadre formel dans lequel le travail est mené.

\chapter{Catalogue de solutions}
Test du catalogue

\chapter{Conclusion}


\vfil
\hspace{8cm}\makeatletter\@author\makeatother\par
\hspace{8cm}\begin{minipage}{5cm}

\end{minipage}

\clearpage
\printbibliography

\appendix
\appendixpage
\addappheadtotoc



\let\cleardoublepage\clearpage
\backmatter

\label{glossaire}
\printnoidxglossary
\label{index}
\printindex

% Le colophon est le dernier élément d'un document qui contient des notes de l'auteur concernant la mise en page et l'édition du document : il est parfaitement optionnel.
\input{colophon.tex}

\end{document}
